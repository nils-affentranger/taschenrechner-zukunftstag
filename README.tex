\documentclass{article}
\usepackage{hyperref}
\usepackage{xcolor}

\title{\textbf{Taschenrechner - Zukunftstag Exanic 2024}}
\date{}

\begin{document}
\maketitle

\section*{\textcolor{teal}{📋 Voraussetzungen}}
Bevor du startest, stelle sicher, dass du Folgendes auf deinem Computer installiert hast:
\begin{itemize}
    \item \href{https://code.visualstudio.com/}{Visual Studio Code}
    \item \href{https://nodejs.org/en/download/prebuilt-binaries/}{Node.js}
    \item \href{https://gitforwindows.org/}{Git}
\end{itemize}

\section*{\textcolor{teal}{🚀 Installation und Start}}

\subsection*{1. Projekt herunterladen}
Öffne die Kommandozeile (CMD) und führe folgende Befehle aus:
\begin{verbatim}
cd C:\Zukunftstag
git clone https://github.com/nils-affentranger/taschenrechner-zukunftstag
\end{verbatim}

\subsection*{2. Projekt in Visual Studio Code öffnen}
\begin{itemize}
    \item Starte Visual Studio Code
    \item Klicke oben links auf "File"
    \item Wähle "Open Workspace from File"
    \item Gehe zu \texttt{C:\Zukunftstag\taschenrechner-zukunftstag}
    \item Suche und wähle die Datei \texttt{taschenrechner-zukunftstag.code-workspace}
\end{itemize}

% Continue with the rest of the steps here

\section*{\textcolor{teal}{🎉 Geschafft!}}
Der Taschenrechner sollte jetzt in deinem Browser erscheinen.

\section*{\textcolor{teal}{❓ Hilfe}}
Falls etwas nicht funktioniert, melde dich. Wir helfen dir gerne weiter!

\hrulefill

\begin{flushright}
Entwickelt für den Zukunftstag bei Exanic - 2024
\end{flushright}

\end{document}
